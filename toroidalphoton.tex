\RequirePackage[l2tabu, orthodox]{nag}
\RequirePackage{fixltx2e}
\documentclass[twocolumn, 10pt, aps, superscriptaddress, floatfix, showpacs, pra, citeautoscript]{revtex4-1}
\usepackage{graphicx}
\usepackage{amsmath}
\usepackage{amssymb}
\usepackage{bm}
\usepackage[utf8]{inputenc}
\usepackage{xcolor}
\usepackage[colorlinks, citecolor={blue!50!black}, urlcolor={blue!50!black}, linkcolor={red!50!black}]{hyperref}
\usepackage{bookmark}
\usepackage{tabularx}
\usepackage{mathtools}
\usepackage{microtype}

\setcounter{secnumdepth}{4}
\setcounter{tocdepth}{4}

\DeclareMathOperator{\e}{e}
\DeclareMathOperator{\de}{d\!}
\DeclareMathOperator{\Tr}{Tr}
\DeclareMathOperator{\diag}{diag}
\DeclareMathOperator{\Res}{Res}
\DeclareMathOperator{\sgn}{sgn}
\DeclareMathOperator{\Det}{Det}
\DeclareMathOperator{\rank}{rank}
\DeclareMathOperator{\im}{Im}
\DeclareMathOperator{\re}{Re}
\newcommand{\vt}[1]{\mathbf{#1}}

\newcommand{\co}[2]{#2}
\renewcommand{\paragraph}{\co}

\DeclarePairedDelimiter\abs{\lvert}{\rvert}%
\DeclarePairedDelimiter\norm{\lVert}{\rVert}%
% Swap the definition of \abs* and \norm*, so that \abs
% and \norm resizes the size of the brackets, and the
% starred version does not.
\makeatletter
\let\oldabs\abs
\def\abs{\@ifstar{\oldabs}{\oldabs*}}
%
\let\oldnorm\norm
\def\norm{\@ifstar{\oldnorm}{\oldnorm*}}
\makeatother

\newcommand{\ev}[1]{\langle#1\rangle}
\newcommand{\bra}[1]{\langle #1|}
\newcommand{\ket}[1]{|#1\rangle}
\newcommand{\bracket}[2]{\langle #1|#2\rangle}

\newcolumntype{L}[1]{>{\raggedright\arraybackslash}p{#1}}
\newcolumntype{C}[1]{>{\centering\arraybackslash}p{#1}}
\newcolumntype{R}[1]{>{\raggedleft\arraybackslash}p{#1}}

\graphicspath{{figures/}}

\begin{document}

\title{Driven-dissipative toroidal Landau levels in cavity arrays}


\author{A. C. Berceanu}
\affiliation{Departamento de F\'isica Te\'orica de la Materia
Condensada \& Condensed Matter Physics Center (IFIMAC), Universidad
Aut\'onoma de Madrid, Madrid 28049, Spain}
\author{Hannah M. Price}
\affiliation{INO-CNR BEC Center and Dipartimento di Fisica,
Universit\`{a} di Trento, I-38123 Povo, Italy}
\author{Tomoki Ozawa}
\affiliation{INO-CNR BEC Center and Dipartimento di Fisica,
Universit\`{a} di Trento, I-38123 Povo, Italy}
\author{Iacopo Carusotto}
\affiliation{INO-CNR BEC Center and Dipartimento di Fisica,
Universit\`{a} di Trento, I-38123 Povo, Italy}

\date{\today}

\begin{abstract}
  We study a driven-dissipative bosonic Harper-Hofstadter model on a
  2D square lattice in the presence of a weak harmonic trap. Recent
  work has anticipated that the eigenstates have the form of momentum
  space toroidal Landau levels. The Landau energy spectrum we observe
  shows signatures of a non-Abelian Berry connection. With an eye to
  optical implementations in cavity arrays with an artificial magnetic
  field for photons, we propose a spectroscopic protocol to
  experimentally characterize the energy and the structure of the
  eigenstates. The feasibility of our proposal using state-of-the-art
  devices is finally assessed.
\end{abstract}

\maketitle


\section{Introduction}

\paragraph{Quantum magnetism is interesting, even more so in momentum space.}
The study of electrons in a planar geometry under the influence of a
strong magnetic field gave rise to such interesting physics as the
integer and fractional quantum Hall effects. With the advent of
synthetic gauge fields, new horizons were opened for simulating these
phenomena with neutral particles, both in an atomic and photonic
context.

\paragraph{Experimental realisations of HH}
The seminal lattice model for exploring this physics is the so-called
Harper-Hofstadter
model~\cite{harper1955magnetic,hofstadter1976butterfly}. This was
recently realized in a multitude of experimental configurations,
ranging from ultracold
gases~\cite{aidelsburger2013hh,miyake2013hh,mancini2015edge,stuhl2015edge},
solid state superlattices~\cite{dean2013hofstadter,yu2014hierarchy}
and silicon photonics~\cite{hafezi2013imaging} to classical systems
such as coupled pendula~\cite{susstrunk2015pendula} and oscillating
circuits~\cite{jia2013circuits}. Even though Hofstadter's original
model dates back to the 70's, much of the afferent physics remained
largely unexplored until recently, due to the very large magnetic
fluxes (of the order of the flux quantum per unit cell) that are
required.
Photonic systems are particularly advantageous for this type of
studies, due to the ability to work at room temperature and the
possibility of local access to the optical field in driven-dissipative
systems~\cite{carusotto2013fluids}. This allows one to directly
observe the effects of such strong artificial gauge fields, by imaging
the photonic wavefunction.

\paragraph{Momentum space dual of HH model provides new insights.}
In Ref.~\onlinecite{price2014magnetic}, the authors considered the
Harper-Hofstadter model on a 2D lattice with the additional presence
of a weak harmonic trapping potential. It was shown that the energy
spectrum is analogous to that of a particle moving in momentum space
on the surface of a torus, in constant magnetic field.  The toroidal
topology arises from the translational symmetry of the model, while
the constant magnetic field is provided by the Berry curvature of the
energy dispersion. Finally, the weak quadratic potential plays the
role of a kinetic energy term in momentum space.

\paragraph{The main goal and its importance.}
The main goal of the current work is to extend the above-mentioned
results to the case of driven-dissipative photonic systems. This would
pave the way for the first direct observation of toroidal Landau
levels in momentum space.
We focus our present study on linear optical systems, such as arrays
of silicon-based coupled ring resonators.  With a proper design of the
resonator array~\cite{hafezi2013imaging}, this latter system has been
demonstrated to be accurately described by the Harper-Hofstadter model
with a strong and uniform artificial gauge field piercing the lattice.
The harmonic trapping potential could then be easily added as a
modulation of the resonator size.


\paragraph{Summary of the manuscript}
After a short theoretical introduction to the model, we present the
main results of numerical simulations. We show that the non-Abelian
Berry connection can have direct physical consequences on the energy
spectrum and finally conclude with a viable proposal for a
photonics-based experiment. A similar theoretical proposal exists for
observing Landau levels on a strained honeycomb lattice in a
driven-dissipative system.~\cite{salerno2015graphene}


\section{Model}
\label{sec:model}

We consider a tight-binding model of non-interacting photons hopping
on a 2D square lattice, with nontrivial hopping phases created, for
instance, by using a synthetic gauge field. We also include an
additional harmonic trapping potential, so one can write the full Hamiltonian as

\begin{equation}\label{eq:model}
\mathcal{H}=\mathcal{H}_0+\frac{1}{2}\kappa
\sum_{m,n}\left[(m-m_0)^{2}+(n-n_0)^{2}\right]\hat{a}_{m,n}^{\dagger}\hat{a}_{m,n}.
\end{equation}

Here $\hat{a}_{m,n}^{\dagger}$ ($\hat{a}_{m,n}$) are the creation
(annihilation) operators for a photon at site $(m,n)$ and $\kappa$ is
the strength of the trap centered at $(m_0, n_0)$. We set the lattice
spacing constant to 1.

$\mathcal{H}_0$ is the Harper-Hofstadter Hamiltonian, the exact form
of which depends on the particular choice of gauge, through the
hopping phases $\phi = (\phi_{m,n}^x, \phi_{m,n}^y)$
%
\begin{multline}\label{eq:hh_hamiltonian}
\mathcal{H}_0=-J\sum_{m,n}(e^{i \phi_{m,n}^x}\hat{a}_{m+1,n}^{\dagger}\hat{a}_{m,n}\\
+e^{i \phi_{m,n}^y}\hat{a}_{m,n+1}^{\dagger}\hat{a}_{m,n}) + \text{h.c.}
\end{multline}
Here $J$ is the real hopping amplitude and the hopping phases $\phi$
are chosen such that the total phase around a plaquette is
$2\pi\alpha$, where $\alpha = \frac{p}{q}$ is the number of magnetic
flux quanta per unit cell ($p$ and $q$ are coprime integers). 

We emphasize that we are dealing with gauge-dependent physics here, as
the eigenstates of $\mathcal{H}$, which can be directly observed in a
photonics-based experiment, depend on the particular choice of
gauge. This is analogous to recent time-of-flight experiments in a
cold atom context~\cite{kennedy2015bec,spielman2011gauge}.
%
In the Landau gauge only the hopping amplitude along one direction is
affected, so we have $\phi = (0, 2\pi\alpha m)$, whereas the symmetric
gauge preserves the $C_4$ rotational invariance of the lattice, so hopping
terms along both $x$ and $y$ are now affected:
$\phi = (-\pi\alpha n, \pi\alpha m)$.

The presence of the constant synthetic gauge field (in real space)
means $\mathcal{H}_0$ is no longer translationally invariant. However,
new translation operators belonging to the magnetic translation
group~\cite{zak1964group, zak1964representations} can be defined. This
results in the so-called magnetic Brillouin zone (MBZ), which is $q$
times smaller than the original BZ.

\subsection{Eigenstates of Hamiltonian}\label{sec:eigenstates}

The eigenfunctions of $\mathcal{H}_0$ are
$\ket{\chi_{n,\vt{p}}} = \frac{1}{\sqrt{N}} e^{i\vt{p}\vt{r}}
u_{n,\vt{p}}$,
where $u_{n,\vt{p}}$ is the periodic Bloch function for band index $n$
and crystal momentum $\vt{p}$, defined in the MBZ, while $N$ is the
number of lattice sites (we take $\hbar = 1$).

We expand the eigenfunctions of the full Hamiltonian in the basis of
$\mathcal{H}_0$ eigenstates
$\ket{\psi} = \sum_n\sum_{\vt{p}} \psi_n(\vt{p})
\ket{\chi_{n,\vt{p}}}$,
and substitute into the Schr\"{o}dinger equation
$i \partial_t \ket{\psi} = \mathcal{H} \ket{\psi}$, obtaining an
equation for the expansion coefficients $\psi_n(\vt{p})$:
\begin{equation}
  i \partial_t \psi_n(\vt{p}) = (\widetilde{\mathcal{H}} \psi)_n(\vt{p})
\end{equation}
where we have introduced the effective momentum-space Hamiltonian
$\widetilde{\mathcal{H}}$. Further algebra (for details,
see~\cite{price2014magnetic}), yields
%
\begin{multline}
  i \partial_t \psi_n(\vt{p}) = E_n(\vt{p}) + \frac{\kappa}{2}\sum_{n^{'},n^{''}}\left(\delta_{n,n^{'}}i \nabla_{\vt{p}} + \mathcal{A}_{n,n^{'}}(\vt{p})\right)\times \\ \times \left(\delta_{n^{'},n^{''}}i\nabla_{\vt{p}} + \mathcal{A}_{n^{'},n^{''}}(\vt{p})\right) \psi_{n^{''}}(\vt{p})
\end{multline}
Here $E_n(\vt{p})$ is the dispersion relation of band $n$ of $\mathcal{H}_0$ and the non-Abelian Berry connection $\mathcal{A}_{n,n^{'}}(\vt{p}) = i\bra{u_{n,\vt{p}}}\nabla_{\vt{p}}\ket{u_{n^{'},\vt{p}}}$. 

Provided the harmonic potential does not significantly mix the energy
bands, one can make a single-band approximation (here we restrict
ourselves to the lowest energy band) of the form
$\psi_{n^{''}}(\vt{p}) = \delta_{0,n^{''}} \psi_0(\vt{p})$, resulting
in 

\begin{equation}\label{eq:dual}
  \widetilde{\mathcal{H}} = E_n(\mathbf{p}) + \delta E_n(\vt{p}) + \frac{\kappa}{2} [i\nabla_{\mathbf{p}} + \mathcal{A}_n(\mathbf{p})]^2 
\end{equation}
with the inter-band Berry connection term
\begin{equation}\label{eq:shift}
  \delta E_n(\vt{p}) = \frac{\kappa}{2}\sum_{n^{'}\neq n} \abs{\mathcal{A}_{n,n^{'}}(\vt{p})}^2.
\end{equation}
This term can be re-written in a more transparent form~\cite{xiao2010berryreview}:
\begin{equation}
  \delta E_n(\vt{p}) = \frac{\kappa}{2}\sum_{n^{'}\neq n} \frac{\abs{\bra{u_{n,\vt{p}}}\left(\nabla_{\vt{p}}\mathcal{H}_0(\vt{p})\right)\ket{u_{n^{'},\vt{p}}}}^2}{\left(E_{n^{'}}(\vt{p}) - E_{n}(\vt{p})\right)^2}
\end{equation}
We now clearly see that if the bandgap is very large, this first order
correction to the energy becomes negligible.


If the bandwidth is much smaller than the trapping energy one can make
a further assumption that the bands are flat, so essentially
$E_n(\vt{p}) = E_n$. The energy bands of $\mathcal{H}_0$ in the case
of $\alpha = 1/q$ become increasingly flat (as compared to the hopping
energy $J$) as $q$ increases.
%
In this limit, the Berry curvature of the first band is also flat over
the whole Brillouin Zone (BZ) and so is $\delta E_n(\vt{p})$ (as
checked by numerics). Therefore, the first 2 terms of
$\widetilde{\mathcal{H}}$ are just a constant shift. The last term has
a transparent physical interpretation. If one thinks of a charged
particle in an electromagnetic field (in real space), then this term
would correspond to the kinetic energy
$\frac{\left[\vt{p} - e\vt{A}(\vt{r})\right]^2}{2M}$.

Within the flat-band approximation, the eigenstates of
Eq.~\eqref{eq:dual} are those of a particle in constant magnetic
field, constrained to move on the surface of a torus, which is the
natural topology of the BZ. The toroidal topology does not influence
the level spacing of the resulting Landau levels, so for fixed
$\alpha$ we get $q$ semi-infinite ladders:
%
\begin{equation}\label{eq:ladders}
  \epsilon_{n,\beta} = E_n + \delta E_n + \left(\beta + \frac{1}{2}\right) \kappa |\Omega_n| 
\end{equation}
where we have introduced the Landau level quantum number
$\beta = 0,1,2,\dots$ and $n = 0 \dots q-1$. One can compare this
formula with the one given in Ref.~\cite{price2014magnetic}, which
does not include the non-Abelian shift contribution. In optical
systems, however, absolute energy is an important quantity, therefore
so is $\delta E_n$.

The Berry curvature
$\vt{\Omega}(\vt{p}) = \nabla_{\vt{p}} \times \mathcal{A}(\vt{p})$
plays the role of momentum space magnetic field and has an average
value $|\Omega_0| = \frac{1}{2\pi\alpha}$. To understand how this
comes about, we note that for odd $q$ and $\alpha = 1/q$, the Chern
number of each band, except the middle one, is -1. Since the Chern
number is the integral over the MBZ of the Berry curvature,
$\mathcal{C}_n = (1/2\pi) \Omega_n A_{\text{MBZ}}$, with
$A_{\text{MBZ}} = (2\pi)^2/q$.
%
Finally, one recognizes $\kappa |\Omega_n|$ as the cyclotron frequency
$\omega_c$.

While the above discussion is generally valid for any translationally
invariant or periodic Hamiltonian $\mathcal{H}_0$, we are dealing with
the concrete case of the Harper-Hofstadter model.

The eigenstates $\chi_\beta$ of $\widetilde{\mathcal{H}}$ in the
natural MBZ for the Landau gauge are known for the lowest energy bands
with $C_n = -1$.

The analytical expression as a function of the Hermite polynomials
$H_\beta$ is~\cite{price2014magnetic}:
%
\begin{multline}\label{eq:chi}
 \chi_\beta (\vt{p}) = \mathcal{N}_\beta^{l_{\Omega_n}} \sum_{j = -
 \infty}^{\infty} e^{- i p_y j a} e^{ - ( p_x + j a l_{\Omega_n}^2 )
 ^2 / 2 l_{\Omega_n}^2}\\
 \times H_\beta ( p_x / l_{\Omega_n} + j a
 l_{\Omega_n})
\end{multline}

Here $l_{\Omega_n} = \sqrt{\frac{1}{|\Omega_n|}}$ is the analogue of
the magnetic length and the normalisation constant is

\begin{equation}
\mathcal{N}_\beta^{l_{\Omega_n}} = \left( \frac{\sqrt{2/q}} {(2^\beta
\beta! \times 2 \pi l_{\Omega_n}^2)} \right)^{1/2}
\end{equation}

The momentum space Berry connection is gauge-dependent and here we
have chosen the Landau gauge,
$\mathcal{A}_n(\mathbf{p}) = \Omega_n p_x \hat{\vt{p}}_y$.


\subsection{Non-Abelian shift}\label{sec:non-abelian-shift}
To quantify the deviations of the exact numerical calculation of the
eigenstates from the eigenspectrum predicted by Eq.~\eqref{eq:ladders}
we introduce two dimensionless parameters: $\eta_{\text{zpe}}$ and
$\eta_{\text{lev}}$. The former quantifies the deviations of the
``zero-point energy" from the predicted value while the latter has to
do with the level spacing within each ladder. For simplicity, we only
look at the first ladder of states.

We are particularly interested in the effect of the non-Abelian term
Eq.~\eqref{eq:shift} on the ``zero-point energy". For the first
ladder, the predicted energy, without this term, is
\begin{equation}
  E_{\text{th}} = \left<E_0(\vt{p})\right>_{\vt{p}} + \frac{1}{2}\frac{\kappa}{2\pi\alpha}
\end{equation}
Here, the symbol $\left<\cdots\right>$ denote an average taken over
the whole MBZ, as for small $q$ the energy spectrum has a momentum
dependence.

The deviation (in units of $\omega_c$) from the exact value
$E_{\text{ex}}$ given by numerical diagonalization of
Eq.~\eqref{eq:model} is
$\eta_{\text{zpe}} = \frac{4\pi\alpha}{\kappa} (E_{\text{ex}} -
E_{\text{th}})$.
Including also the non-Abelian contribution, we define
$\eta_{\text{zpe}}^{\text{nab}} = \frac{4\pi\alpha}{\kappa}
(E_{\text{ex}} - E_{\text{th}} - \left<\delta
  E_1(\vt{p})\right>_{\vt{p}})$.
We plot both quantities in the top panel of Fig.~\ref{fig:zpe}. In the
same panel we plot $\eta_{\text{zpe}}^{\text{nab}}$ twice, first
considering the whole sum in Eq.~\eqref{eq:shift} and then restricting
it to just the first term, which represents the shift due to the
second band only. The difference between the 2 curves is negligible,
indicating that this term is the dominant one.  We see that the
non-Abelian contribution is indeed crucial to the ``zero-point
energy", acting as a uniform positive shift on the whole ladder of
states and bringing it closer to the exact value. We emphasize that
this leads to a straight-forward way of measuring a non-Abelian
quantity in a topologically nontrivial system. We note that a proposal
for measuring non-Abelian effects in ultracold gases already
exists~\cite{Grusdt2014nonabelian}, however it is based on an
interferential scheme, while our proposal is observable as a direct
effect in the system dynamics.

\begin{figure}[htb]\centering
  \includegraphics[width=0.9\linewidth]{nonabcorr} % anc/scripts/nonabelian.jl
  \caption{\emph{Top}: ``Zero-point energy" error, with (cont. curve,
    $\eta_{\text{zpe}}^{\text{nab}}$) and without (dashed line,
    $\eta_{\text{zpe}}$) the non-Abelian shift. Including just the
    first term in the sum Eq.~\eqref{eq:shift} gives an identical
    curve to the one of $\eta_{\text{zpe}}^{\text{nab}}$. The chosen
    trap strength is $\kappa = 0.02 J$.  \emph{Bottom}: Level-spacing
    error, for the same trap strength, considering the lowest 2 levels
    only ($\beta = 0,1$, cont. curve) or $\beta = 1,2$ (dashed line).}
  \label{fig:zpe}
\end{figure}

In the bottom panel of Fig.~\ref{fig:zpe} we look at the error of the
level-spacing between the first 2 ladder states,
$\eta_{\text{lev}} = \frac{2\pi \alpha}{\kappa} [E_{\text{ex}}(\beta =
1) - E_{\text{ex}}(\beta = 0)] -1$.
The level-spacing error is not affected by the non-Abelian shift and
instead originates from band-mixing effects due to the breakdown of
the single-band approximation. However, even if we used an effective
model we see that $\eta_{\text{lev}} \ll 1$.

It is a well-known result of perturbation theory that band-mixing
induced by an external potential acts as a negative shift on the lower
energy band.  The bandgap decreases as roughly $\frac{1}{q}$, thus
facilitating band-mixing at higher values of $q$. We can in fact see a
similar trend also in the upper panel, for large $q$.

Summarising, we can distinguish 2 sources of error from
Fig.~\ref{fig:zpe}: one coming in at small $q$ and being due to the
breakdown of the flat-band approximation and the other one at large
$q$, due to band mixing caused by the diminishing band gap.




\section{Driving and dissipation}\label{sec:driven-dissipation}

Following the treatment of Ref.~\onlinecite{ozawa2014qhe}, we now
include dissipation in the model by assuming uniform and local losses
characterized by the loss rate $\gamma$. The pump is assumed to be
monochromatic, with frequency $\omega_0$ and spatial profile
$f_{m,n}$.  Furthermore, as we are dealing with a noninteracting
system, we can safely replace the operators with their expectation
values and assume that their steady state evolution follows that of
the pump: $a_{m,n}(t) = a_{m,n} e^{-i \omega_0 t}$.

One then needs to solve the resulting Heisenberg equation of motion:
%
\begin{equation}
i\partial_{t}a_{i,j}(t)+i\gamma
a_{i,j}(t)-f_{i,j}e^{-i\omega_{0}t}=\left[a_{i,j}(t),\mathcal{H}\right]
\end{equation}
resulting in a set of linear equations to be solved numerically for each
lattice site:
%
\begin{multline}\label{eq:linear_problem}
J[e^{-i\phi_{m,n}^x}a_{m+1,n}+e^{i\phi_{m,n}^x}a_{m-1,n}+\\
+e^{-i\phi_{m,n}^y}a_{m,n+1}+e^{i\phi_{m,n}^y}a_{m,n-1}]+\\
+\left[\omega_{0}+i\gamma-\frac{1}{2}\kappa
\left((m-m_0)^{2}+(n-n_0)^{2}\right)\right]a_{m,n}=f_{m,n}
\end{multline}

The expectation values $\abs{a_{m,n}}^2$ give us the number of photons
at site $(m,n)$, whereas the intensity spectrum is given by their
total sum $\sum_{m,n} |a_{m,n}|^2$ as a function of pump frequency
$\omega_0$. Filtering the near(far)-field emission at a particular
frequency $\omega$ gives us a reconstruction of the
real(momentum)-space wavefunction of $\widetilde{\mathcal{H}}$ at the
corresponding energy. Finally, one has to fold the far-field emission
from the full BZ to the MBZ.

\section{Results and discussion}
\label{sec:results}

We now present the main results of numerically solving
Eq.~\eqref{eq:linear_problem} for a lattice of
$N \times N = 45 \times 45$ sites, with $\kappa = 0.02 J$,
$\gamma = 0.001 J$ and $\alpha = 1/11$.  The numerical code was written
in Julia~\cite{bezanson2014julia} and is available online as ancillary
files for this preprint.

For simplicity we limit ourselves to pump frequencies which lie
withing the first ladder of states, given by Eq.~\eqref{eq:ladders}
for the case of $n = 0$.


\subsection{Selection rules}
\label{sec:selection}

\paragraph{The overlap between the pump state and the eigenstate determines the intensity.}
We first look at the intensity spectrum as a function of pump
frequency Fig.~\ref{fig:pumping_schemes}.  Whenever the pump frequency
$\omega_0$ is resonant with an eigenstate $\epsilon_{0,\beta}$ of the
Hamiltonian Eq.~\eqref{eq:model}, we should get a peak in the
spectrum.

In fact, a general result~\cite{carusotto2013fluids} is that the
different eigenmodes of a driven-dissipative system appear as peaks in
the transmission and/or absorption spectra under a coherent pump.
When the pump frequency is set on resonance with a given mode, the
intensity profiles in both real- and momentum-space reproduce that
mode's wavefunction.


The peaks are of course broadened by the decay rate $\gamma$, and
their height depends on the overlap between the spatial amplitude
profile of the pump and the underlying eigenstate of $\mathcal{H}$ at that
energy. Furthermore, as we are dealing with
gauge-dependent effects, the particular choice of gauge will also
affect the spectrum. To exemplify, we compare 4 different pumping
schemes in both the Landau and symmetric gauge.
%

\begin{figure*}[htb]\centering
  \includegraphics[width=\linewidth]{selection} % anc/scripts/selection_fig.jl
  \caption{Intensity spectra for different pumping conditions: (a)
    pumping the single site (5,5), (b) pumping with a Gaussian
    profile, (c) homogeneous pumping and (d) pumping with a random
    phase. Black curves correspond to using the Landau gauge while
    green (dashed) ones to the symmetric gauge. The dotted vertical
    lines (with labels indicating the value of $\beta$) mark the
    states which were selected for later analysis. The spectra in
    panels (a) and (d) are identical for both gauges.}
  \label{fig:pumping_schemes}
\end{figure*}

We must emphasize that, while the absolute value of the real-space
wavefunction is identical in both gauges, its phase is not, and
neither is its momentum space representation (see
Fig.~\ref{fig:hom_mom_sp}). It is only the symmetric gauge which has a
well defined $90^{\circ}$ rotational symmetry. This is important when
examining the selection rules.

\paragraph{A δ pump in real space is homogeneous in momentum space.}
The simplest case is that of pumping a single site, with a pump of the
form $f_{m,n} = \delta_{m,m_0} \delta_{n,n_0}$. The pump is of course
homogeneous in momentum space, so one must think of real space when
explaining the shape Fig.~\ref{fig:pumping_schemes}, panel (a).
Indeed, the wavefunctions of $\mathcal{H}$ in real space are rings of finite
width which increase in radius as the energy increases (see
Fig.~\ref{fig:delta_real_sp}). One can in fact deduce that the radius
of the real space rings scales as $r^2 \approx \frac{1}{\pi} q
\beta$.
To see this, one can think of the Landau level energy
$(\beta + 1/2)\omega_c$ as the momentum-space kinetic energy
$\frac{\kappa}{2}r^2$, with
$r = i\nabla_{\mathbf{p}} + \mathcal{A}_1(\mathbf{p})$ being the
``physical'' position operator.
\begin{figure}[htb]
  \centering
  \includegraphics[width=0.9\linewidth]{real} % anc/scripts/selection_fig.jl
  \caption{Real space reconstruction of the states $\beta=3$, 6, 15
    and 26, corresponding to panel (a) of
    Fig.~\ref{fig:pumping_schemes}.}
  \label{fig:delta_real_sp}
\end{figure}


Therefore, if one pumps an off-center site, there will only be a
limited range of rings that will have radii such as to overlap with
the pump spot and become visible. For $r=5\sqrt{2}$, we get
$\beta \approx 14$, in good agreement with
Fig.~\ref{fig:pumping_schemes} (a). Changing the gauge is equivalent
to changing the relative phase between different sites, so since we
are pumping a single site, the result in the Landau gauge will be identical.


%
\begin{figure}[htb]\centering
  \includegraphics[width=0.9\linewidth]{momentum} % anc/scripts/selection_fig.jl
  \caption{Momentum space reconstruction of the
    eigenstates. \emph{Top}: The states corresponding to $\beta=0$, 2,
    4 and 6 in panel (c) of Fig.~\ref{fig:pumping_schemes}, using
    Landau gauge.  \emph{Bottom}: The states corresponding to
    $\beta=0$, 1, 9 and 20 in panel (b) of
    Fig.~\ref{fig:pumping_schemes}, using symmetric gauge.}
  \label{fig:hom_mom_sp}
\end{figure}


\begin{figure}[htb]
  \centering
  \includegraphics[width=0.9\linewidth]{sym_ring} % anc/torus_edge.ipynb
  \caption{Wavefunction in the BZ, using the symmetric
    gauge. \emph{Top}: The states corresponding to
    $\beta = 9, 20, 30$.
    \emph{Bottom}:  The states corresponding to
    $\beta = 38, 59, 99$.}
  \label{fig:torus_edge}


\end{figure}

\begin{figure}[htb]
  \centering
  \includegraphics[width=0.9\linewidth]{fringe_trap} % anc/torus_edge.ipynb
  \caption{Wavefunction in the BZ, using the Landau gauge. We have
    considered the state $\beta = 4$ for different trap
    positions (from left to right): $m_0 = 0$ (trap in the
    center), 2, 5.5 and 11.}
  \label{fig:moving_trap}
\end{figure}

\paragraph{Gaussian pumping is now bound also in momentum space.}
The next logical step in complexity is to use a Gaussian pump, a more
extended version of the $\delta$ pump. Its amplitude is given by
$f_{m,n} = \exp- \frac{1}{2\sigma^2} \left[(m-m_0)^2 + (n-n_0)^2
\right]$.
We take one which is also centered at $(m_0,n_0) = (5,5)$, with
$\sigma =1$. The main effect is, as expected, that now more states
become visible in both the low (smaller $\beta$) and high-energy
(larger $\beta$) sections of the spectrum, with a dip in the
intermediate region caused by gauge-dependent interference
effects. The intensity of the states however is more difficult to
explain, due to the overlap with pump in momentum space. For example,
as the symmetric gauge states have a ring-like structure also in
momentum space (see bottom panel of Fig.~\ref{fig:hom_mom_sp}), the
high energy portion of the spectrum is washed out compared to its
Landau gauge counterpart. The white spot close to the edges of the
rings in Fig.~\ref{fig:hom_mom_sp} is due to the destructive
interference with the pump.

In the symmetric gauge, real-space wavefunctions have a phase which
winds around the ring as $e^{i\beta \phi}$. The shortest lengthscale
in real space corresponds to this phase winding, and its inverse
corresponds to the radius of the rings in momentum space. We can thus
deduce that $p^2 = \pi \frac{\beta}{q}$, as seen also in
Fig.~\ref{fig:hom_mom_sp}, bottom panel. For large values of $\beta$
the rings start touching the BZ torus boundaries. We show such an
example in Fig.~\ref{fig:torus_edge}, where one can see the pattern
created by the self-interference of the wavefunction at the torus
edge.

Another interesting effect is that of the trap position on the
wavefunction in the BZ. As shown in~\cite{ozawa2014momhh}, moving the
harmonic trap changes the boundary conditions on the BZ torus. An
exemplification of that is Fig.~\ref{fig:moving_trap}, showing that
moving the trap in the positive $x$ direction shifts the observed
momentum-space pattern vertically, with a periodicity of $q$.


\paragraph{Homogeneous spectrum can be understood using a parity argument in momentum space.}
The limit of a very wide Gaussian is a homogeneous pump profile, and
that is what we consider in panel (c) of
Fig.~\ref{fig:pumping_schemes}. Now $f_{m,n} = f$, and the situation
looks very different in the 2 gauges.  In the Landau gauge we see only
half of the states. The reason for this particular selection rule
becomes apparent as one plots the eigenstates in the full BZ, as shown
in the top row of Fig.~\ref{fig:hom_mom_sp}. The states with an odd
value of $\beta$ have an odd number of nodes, with one node in the
center. This also has an analytical justification given by the
properties of the Hermite polynomials in Eq.~\eqref{eq:chi}. Therefore
these states do not overlap with a homogeneous pump, which is a
$\delta$ function in momentum space, centered in the middle of the BZ.
%

The symmetric gauge, on the other hand, shows one out of every 4
states, as can be seen in the inset of Fig.~\ref{fig:pumping_schemes}
(c). This is due to the fact that, on a square lattice, the angular
momentum is conserved modulo 4, respecting the 4-fold rotational
symmetry. The peak intensity gets smaller for larger $\beta$ because
of the diminishing overlap with the increasing momentum-space ring.

\paragraph{One must use a random phase in order to see all the states.}
In order to see the complete ladder of states, we chose a pump which
has a random site-dependent phase $\phi_{m,n}$:
$f_{m,n}=fe^{i\phi_{m,n}}$.  The phases are chosen from a random
uniform distribution, and have values in the interval $[0,2\pi)$. The
bottom panel of Fig.~\ref{fig:pumping_schemes} was obtained by
averaging over 100 distinct realizations of these random phases. That
results in a relatively even intensity distribution, for both gauges.

\section{Experimental proposal}
\label{sec:experiment}

\paragraph{Greater dissipation implies spectral broadening.}
We now focus on a realistic system, within experimental reach. The
lattice size is $11 \times 11$ sites and the new value of the loss
rate is $\gamma = 0.05 J$, in the same range as the losses in
Ref.~\onlinecite{hafezi2013imaging}. This of course broadens the
levels, making them harder to resolve. Looking at
Eq.~\eqref{eq:ladders}, we see that the level spacing is given by
$\frac{\kappa}{2\pi\alpha}$, so in order to increase the spacing we
must consider a stronger trap. We chose $\kappa = 0.2 J$ and
$\alpha = 1/7$. The first band of $\mathcal{H}_0$ is still relatively
flat compared to the hopping $J$ for this value of $\alpha$ and the
bandwidth $E_2 - E_1$ is larger than before, preventing the band
mixing due to the stronger trap potential.

We work in the Landau gauge and pump a single site, situated on the
upper border of the system, at $(m_0,n_0)= (0,5)$. This closely
resembles the pumping scheme used in optical experiments such as the
one of Ref.\cite{hafezi2013imaging}. 

\begin{figure}[htb]
  \centering
  \includegraphics[width=0.9\linewidth]{exp_spect} % anc/scripts/experimental_fig.jl
  \caption{\emph{Top}: Intensity spectrum for $N=11$ and
    $\gamma = 0.05$. We filter at the energy of the eigenstate
    $\beta=7$ (long dashed line). The orange vertical dashed lines
    show the eigenstates of $\mathcal{H}$ from Eq.~\eqref{eq:model}.
    \emph{Bottom}: Slice along the $p_x^0 = 0$ line in the MBZ
    (continuous, black) compared to the analytical prediction of
    Eq.~\eqref{eq:chi}, $|\chi_7(0,p_y)|^2$ (dotted, blue). The orange
    dashed curve shows the same slice for the non-dissipative case.}

  \label{fig:exp_spectrum}
\end{figure}

The spectrum is shown in Fig.~\ref{fig:exp_spectrum}. Apart from the
expected broadening due to larger losses, we note that for frequencies
larger than $-1.5 J$ we start to see states from the second ladder
$\epsilon_{1,\beta}$ (see Eq.~\eqref{eq:ladders}). Their proximity to
the first ladder states means they cannot be resolved as separate
peaks in the dissipative spectrum.

We filter at the energy indicated by the black dashed line and plot
the resulting emission in real and momentum space (full BZ), as well
as in the MBZ. For comparison, we also plot the wavefunction of the
Hamiltonian $\mathcal{H}$ in Eq.~\eqref{eq:model} as the top row of
Fig.~\ref{fig:exp_states}. The difference between the two is of
course, the lack of pumping and dissipation of the latter. We see
that, even in the presence of finite size effects due to such a small
system, one can still distinguish the nodal structure of the emission
in momentum space, characteristic of a toroidal Landau level.  One can
also notice that the bottom left panel of Fig.~\ref{fig:exp_states}
shows a decaying cyclotron orbit in momentum space, with $\gamma$
playing the role of inverse lifetime. In the same way that real-space
Landau levels give rise to real-space cyclotron orbits, the
momentum-space Landau levels can provide the first direct observation
of a cyclotron orbit in momentum space.

Furthermore, as the expression of the wavefunction in MBZ for the
Landau gauge is given by Eq.~\eqref{eq:chi}, we can compare by taking
a slice along the $p_x^0 = 0$ line in the third column of
Fig.~\ref{fig:exp_states}. The qualitative structure in 2D is similar,
as confirmed by the cuts along the vertical dotted lines, shown in the
botttom panel of Fig.~\ref{fig:exp_spectrum}.


\begin{figure}[htb]
  \centering
  \includegraphics[width=0.9\linewidth]{experimental} % anc/scripts/experimental_fig.jl
  \caption{\emph{Top row}: Emission of the $\beta=7$ state in real
    (left column), momentum-space (middle column) and MBZ (right
    column) with no pumping or decay.
    \emph{Bottom row}: Emission from the state at $\omega_0 = -1.63 J$.
    The $\delta$-like pump at (0,5) is visible as a dark square.}

  \label{fig:exp_states}
\end{figure}

One a final note, we briefly describe how one can measure the
non-Abelian contribution $\delta E_0$ and hence the non-Abelian Berry
connection (see Eq.~\eqref{eq:shift}). Starting from an experimental
spectrum, one first needs to select a particular peak and determine
its $\beta$ label by comparing the MBZ reconstruction with the
analytical result. The distance between two neighbouring peaks gives
the level spacing $\kappa \abs{\Omega_0}$. Finally, to separate the
non-Abelian shift $\delta E_0$ from the Harper-Hofstadter ground state
energy $E_0$ in Eq.~\eqref{eq:ladders}, one can make use of the fact
that the former depends on the trap strength $\kappa$, while the
latter does not. Preparing 2 different samples with different trap
strengths and substracting the ground state energy thus allows a
direct measure of $\delta E_0$.

\section{Conclusion}
\label{sec:conclusion}

% higher level of analysis
% significance of the work

In conclusion, we have shown that toroidal Landau levels in momentum
space can readily be observed in linear driven-dissipative photonic
systems. Currently existing setups, such as the arrays of
silicon-based coupled ring resonators of
Ref.~\onlinecite{hafezi2013imaging} would make a prime candidate for
such an experiment.

\paragraph{One possible outlook is to include interactions.}
One interesting outlook would be to include the effect of photon-photon interactions in the model. When the synthetic gauge field is combined with strong interactions, one can hope to observe the hallmarks of the fractional quantum Hall physics.


\acknowledgments

ACB acknowledges financial support from the ESF through the POLATOM grant 4914.

HMP, TO and IC acknowledge financial support by the Autonomous Province of Trento, partially through the Call ``Grandi Progetti 2012", Project ``On silicon chip quantum optics for quantum computing and secure communications - SiQuro" and by the ERC through the QGBE grant.

\bibliographystyle{apsrev4-1}
\bibliography{topo}


\end{document}
