\RequirePackage[l2tabu, orthodox]{nag}
\RequirePackage{fixltx2e}
\documentclass[twocolumn, 10pt, aps, superscriptaddress, floatfix, showpacs, prb, citeautoscript]{revtex4-1}
\usepackage{graphicx}
\usepackage{amsmath}
\usepackage{amssymb}
\usepackage{bm}
\usepackage[utf8]{inputenc}
\usepackage{xcolor}
\usepackage[colorlinks, citecolor={blue!50!black}, urlcolor={blue!50!black}, linkcolor={red!50!black}]{hyperref}
\usepackage{bookmark}
\usepackage{tabularx}
\usepackage{mathtools}
\usepackage{microtype}

\setcounter{secnumdepth}{4}
\setcounter{tocdepth}{4}

\DeclareMathOperator{\e}{e}
\DeclareMathOperator{\de}{d\!}
\DeclareMathOperator{\Tr}{Tr}
\DeclareMathOperator{\diag}{diag}
\DeclareMathOperator{\Res}{Res}
\DeclareMathOperator{\sgn}{sgn}
\DeclareMathOperator{\Det}{Det}
\DeclareMathOperator{\rank}{rank}
\DeclareMathOperator{\im}{Im}
\DeclareMathOperator{\re}{Re}
\newcommand{\vt}[1]{\mathbf{#1}}

\newcommand{\co}[2]{#2}
\renewcommand{\paragraph}{\co}

\DeclarePairedDelimiter\abs{\lvert}{\rvert}%
\DeclarePairedDelimiter\norm{\lVert}{\rVert}%
% Swap the definition of \abs* and \norm*, so that \abs
% and \norm resizes the size of the brackets, and the
% starred version does not.
\makeatletter
\let\oldabs\abs
\def\abs{\@ifstar{\oldabs}{\oldabs*}}
%
\let\oldnorm\norm
\def\norm{\@ifstar{\oldnorm}{\oldnorm*}}
\makeatother

\newcommand{\ev}[1]{\langle#1\rangle}
\newcommand{\bra}[1]{\langle #1|}
\newcommand{\ket}[1]{|#1\rangle}
\newcommand{\bracket}[2]{\langle #1|#2\rangle}

\newcolumntype{L}[1]{>{\raggedright\arraybackslash}p{#1}}
\newcolumntype{C}[1]{>{\centering\arraybackslash}p{#1}}
\newcolumntype{R}[1]{>{\raggedleft\arraybackslash}p{#1}}

\graphicspath{{figures/}}

\begin{document}

\title{Proposal for observation of toroidal Landau levels in driven dissipative photonic systems}


\author{A. C. Berceanu}
\affiliation{Departamento de F\'isica Te\'orica de la Materia
Condensada \& Condensed Matter Physics Center (IFIMAC), Universidad
Aut\'onoma de Madrid, Madrid 28049, Spain}
\author{Hannah M. Price}
\affiliation{INO-CNR BEC Center and Dipartimento di Fisica,
Universit\`{a} di Trento, I-38123 Povo, Italy}
\author{Tomoki Ozawa}
\affiliation{INO-CNR BEC Center and Dipartimento di Fisica,
Universit\`{a} di Trento, I-38123 Povo, Italy}
\author{Iacopo Carusotto}
\affiliation{INO-CNR BEC Center and Dipartimento di Fisica,
Universit\`{a} di Trento, I-38123 Povo, Italy}

\date{\today}

\begin{abstract}
  We propose an experimental scheme for the observation of toroidal
  Landau levels, in momentum space, using a driven dissipative
  photonic system.
\end{abstract}

\maketitle



\section{Introduction}

\paragraph{First paragraph.}
The main objective of this work was to extend the formalism introduced
in Ref.~\onlinecite{price2014magnetic} to the case of a driven
dissipative system. In light of the recent surge in popularity of the
study of topological effects in photonic systems, we propose an
experimental scheme for observing Landau levels in momentum space in
such a system.


\section{Model}
\label{sec:model}

\subsection{Harper-Hofstadter Hamiltonian (no trap)}
\label{sec:hh_notrap}

Without the harmonic trap, we have the celebrated Harper-Hofstadter
Hamiltonian (unless otherwise specified, we work in the Landau gauge):

\begin{equation}\label{eq:gigi}
  \mathcal{H}_0=-J\sum_{m,n}(\hat{a}_{m+1,n}^{\dagger}\hat{a}_{m,n}+e^{i2\pi\alpha m}\hat{a}_{m,n+1}^{\dagger}\hat{a}_{m,n}) + \text{h.c.}
\end{equation}

In order to solve this Hamiltonian, the $q \times q$ matrix one has to diagonalize is

\begin{equation}
-J\begin{pmatrix}
    v_1 & 1 & 0 & 0 & \cdots & e^{-iqp_y}\\
    1 & v_2 & 1 & 0 & \cdots & 0\\
    \cdots & \cdots & \cdots & \cdots & \cdots & \cdots\\
    0 & 0 & \cdots & 1 & v_{q-1} & 1\\
    e^{i q p_y} & 0 & \cdots & 0 & 1 & v_q\\
  \end{pmatrix}
\end{equation}

where $v_j = 2\cos(p_x^0 + 2\pi \alpha j)$, with $j$ running from 1 to $q$.

\subsection{Harmonic trapping of a dissipative system}
\label{sec:hh_trap}

In the presence of a harmonic trap, one needs to solve Hamilton's
equation of motion for the following Hamiltonian:

\begin{equation}
H=\mathcal{H}_0+\frac{1}{2}\kappa
a^{2}\sum_{m,n}(m^{2}+n^{2})\hat{a}_{m,n}^{\dagger}\hat{a}_{m,n}
\end{equation}

Introducing pumping and decay, Hamilton's equation reads
\begin{equation}
i\partial_{t}a_{i,j}(t)+i\gamma
a_{i,j}(t)-f_{i,j}e^{-i\omega_{0}t}=\left[a_{i,j}(t),H\right]
\end{equation}

Expressing $\omega_{0}$, $\gamma$, $\kappa$ and $f$ in units of $J$
and setting $a=1$ we get

\begin{multline}
a_{m+1,n}+a_{m-1,n}+e^{-i2\pi\alpha ma}a_{m,n+1}+e^{i2\pi\alpha
ma}a_{m,n-1}\\
+\left[\omega_{0}+i\gamma-\frac{1}{2}\kappa
a^{2}(m^{2}+n^{2})\right]a_{m,n}=f_{m,n}
\end{multline}

This has a similar form to Eq.~(2) of Ref.~\onlinecite{ozawa2014qhe}.

\subsection{Magnetic Brillouin zone}
In the MBZ, the analytical expression for an eigenstate labeled $\beta$ (with
$\beta= 0,1,2,\dots$) reads

\begin{multline}
 \chi_\beta ( {\bf p}) = \mathcal{N}_\beta^{l_{\Omega_n}} \sum_{j = -
 \infty}^{\infty} e^{- i p_y j a} e^{ - ( p_x + j a l_{\Omega_n}^2 )
 ^2 / 2 l_{\Omega_n}^2}\\
 \times H_\beta ( p_x / l_{\Omega_n} + j a
 l_{\Omega_n})
\end{multline}

and the normalisation constant is

\begin{equation}
\mathcal{N}_\beta^{l_{\Omega_n}} = \left( \frac{\sqrt{2/q}} {(2^\beta
\beta! \times 2 \pi l_{\Omega_n}^2)} \right)^{1/2}
\end{equation}

Here $H_\beta$ are the Hermite polynomials, $l_{\Omega_n} =
\sqrt{\frac{1}{|\Omega_n|}}$ is the characteristic momentum scale,
with the average Berry curvature $|\Omega_n| = \frac{a^2}{2\pi\alpha}$.

\subsection{Symmetric gauge}
So far we have used the Landau gauge, which preserves translational
invariance along the $y$-axis: $A = (0, Bx)$. In the Landau gauge,
only the hopping amplitude along $y$ is affected.  But one can also
use the symmetric gauge, which preserves rotational invariance: $A =
\frac{B}{2}(-y, x)$. Hopping terms along both $x$ and $y$ are now
affected.  The new Hamiltonian is
\begin{multline}
\mathcal{H}_0=-J\sum_{m,n}(e^{-i\pi\alpha
n}\hat{a}_{m+1,n}^{\dagger}\hat{a}_{m,n}\\
+e^{i\pi\alpha
m}\hat{a}_{m,n+1}^{\dagger}\hat{a}_{m,n}) + \text{h.c.}
\end{multline}

\section{Results and discussion}
\label{sec:results}

In the following plots, if not otherwise specified, we use a system of $N \times N = 45 \times 45$ sites, with $\kappa =
0.02$, $\gamma = 0.001$ and $\alpha = 1/11$.

\subsection{Selection rules}
\label{sec:selection}

The visibility of a state in the intensity vs pump frequency spectrum
is determined by the overlap between the pump state and the eigenstate
of the system at that frequency.  To exemplify, we compare 4 different
pumping schemes.

\begin{figure}[htb]
  \centerline{\includegraphics[width=\linewidth]{selection}}
  \caption{Intensity spectra for different pumping conditions: (a)
    homogeneous pump, (b) pumping with a random phase, (c) pumping a
    single site and (d) a pump with a Gaussian profile. The dashed
    orange lines are the positions of the exact eigenstates for an
    equilibrium system.}
  \label{fig:pumping_schemes}
\end{figure}



\paragraph{Random phase}
The pumping term in this scheme is given by
$f_{m,n}=fe^{i\phi_{m,n}}$. The phases $\phi_{m,n}$ are chosen
from a random nomal distribution, and have values in the interval
$[0,2\pi)$ . Using this scheme, all states become visible.

\paragraph{Homogeneous pumping}
In the case of homogeneous pumping, $f_{m,n} = f$. Here we see only
half the states (in the Landau gauge, that is). The reason for this
particular selection rule becomes apparent as one plots the
eigenstates in momentum space. We see that the states with an odd
value of $\beta$ have an odd number of nodes, with one node in the
center (this can be understood from the properties of the Hermite
polynomials in the analytical expression in the MBZ). Therefore the
overlap between this set of states and a homogeneous pump, which is a
$\delta$ function in momentum space centered at (0,0) is zero.

\begin{figure}[htb]
  \centerline{\includegraphics[width=0.9\linewidth]{momentum}}
  \caption{Momentum space representation of the eigenstates
    corresponding to $\beta=1$ (a), $\beta=3$ (b) and $\beta=5$ (c).}
  \label{fig:hom_mom_sp}
\end{figure}



\paragraph{$\delta$-like pump}
If one pumps a single site, using a pump of the form $f_{m,n} =
\delta_{m,m_0} \delta_{n,n_0}$, the pump profile will be homogeneous
in momentum space.  However, one can now understand the associated
selection rule by looking at the overlap in real-space. The real-space
eigenstates are concentric circles with increasing radii (as one goes
up the ladder of states). Therefore, is one pumps an off-center site,
there will only be a small range of states, with radii close to
distance between the origin and the pumped site, which will have
considerable overlap with the pump. Those are the states that become
visible.

\begin{figure}[htb]
  \centerline{\includegraphics[width=0.9\linewidth]{real}}
  \caption{Real space representation of the states $\beta=4$ (a), $\beta=9$ (b) and $\beta=14$ (c). The single site (5,5) was pumped.}
  \label{fig:delta_real_sp}
\end{figure}


\paragraph{Gaussian pump}
Last but not least, one can use a more extended, Gaussian pump, given
by \(f(x,y) = A \exp- \frac{1}{2\sigma^2} \left[(x-x_0)^2 + (y-y_0)^2
\right]\). We take one which is also centered at \((m_0,n_0) = (5,5)\),
with \(\sigma =1\) . The main effect is that now the range of visible
states becomes extended in both directions of smaller radii (smaller
\(\beta\)) and larger radii (larger \(\beta\)).

\subsection{Gauge dependence}
\label{sec:gauge}
It is important to realise that we are dealing with gauge-dependent
physics here. For instance, if one uses a symmetric gauge instead of
the Landau one, the selection rules are modified. Now the momentum
space representation of the eigenstates are also concentric rings with
increasing radii. So with a homogeneous pump, only the lowest
eigenstate $\beta=0$ is visible because for higher states the ring
opens in momentum space. The random phase and single-site pumping
spectra show no difference from the Landau gauge results, while the
Gaussian pump result now becomes a shifted version of the
single-site-pump result.



\section{Experimental proposal}
\label{sec:experiment}

We now focus on a realistic system, with $11 \times 11$ sites. The new
value of the dissipation constant is now closer to experiments,
$\gamma = 0.0505$. This makes the states harder to resolve, so we must
also take a stronger trap to increase the level spacing. We chose
$\kappa = 0.2$ and $\alpha = 1/7$. We work in the Landau gauge and
pump the single site situated on the upper edge of the system, at
$(m_0,n_0)= (0,5)$.

\begin{figure}[htb]
  \centerline{\includegraphics[width=0.9\linewidth]{exp_spect}}
  \caption{\emph{Top}: Intensity spectrum for $N=11$ and $\gamma = 0.0505$. We filter at the energy of the eigenstate $\beta=7$ (long red dashed line).
\emph{Bottom}: Slice along the $p_x^0 = 0$ line in the MBZ (cont line, black) compared to the analytical prediction (dotted line, red). The orange dashed line shows the same slice for the non-dissipative case.}
  \label{fig:exp_spectrum}
\end{figure}


\begin{figure}[htb]
  \centerline{\includegraphics[width=0.9\linewidth]{experimental}}
  \caption{\emph{Bottom row}: Emission from the state at $\omega_0 =
-1.63$ in real (left column), momentum-space (middle column) and MBZ
(right column). The $\delta$-like pump at (0,5) is visible as a black
square.  \emph{Top row}: Emission at the same energy, but with no
pumping or decay.}
  \label{fig:exp_states}
\end{figure}




\acknowledgments

ACB acknowledges financial support from the ESF through the POLATOM grant 4914.

HMP, TO and IC acknowledge financial support by the Autonomous Province of Trento, partially through the Call ``Grandi Progetti 2012", Project ``On silicon chip quantum optics for quantum computing and secure communications - SiQuro" and by the ERC through the QGBE grant.

\bibliographystyle{apsrev4-1}
\bibliography{topo}

\appendix

\section{Approximations}\label{app:approximations}
The two main approximations are the flat-band approximation, where one
neglects the momentum dependence of the lowest HH band and the weak
potential approximation which implies that the trap is not strong
enough to cause significant mixing with the other energy bands.  The
HH Hamiltonian has $q$ bands, which get exponentially flat in the
limit of large $q$. Quantitatively, we have that the bandwidth $BW =
E_1^{\text{max}} - E_1^{\text{min}} = A e^{b q}$ , with $b = -1.09$
and $A = 7.4$ . The bandgap however gets smaller with increasing $q$,
increasing the possibility of band mixing. We have that $\Delta E =
E_2 - E_1 = \frac{A}{q}$ , with $A=9.63$ .


\begin{figure}[htb]
  \centerline{\includegraphics[width=0.7\linewidth]{HH/hh.png}}
  \caption{Bandwidth and bandgap behaviour for the HH model (no trap).}
  \label{fig:energy_bands}
\end{figure}


One can compare the theoretical expression for the ladder of states labeled by $\beta$ 
to the ``real" energies given by exact diagonalization

\begin{subequations}
  \begin{align}
    E_T &= E_1 + \frac{1}{2}\frac{\kappa}{2\pi\alpha} + \beta \frac{\kappa}{2\pi\alpha}\\
    E_R &= E_1 +
          (1+\eta_{\text{ZPE}})\frac{1}{2}\frac{\kappa}{2\pi\alpha} +
          (1+\eta_{\text{L}})\beta \frac{\kappa}{2\pi\alpha}
  \end{align}   
\end{subequations}
  


Here we have introduced two parameters to quantify the observed
differences. One is the ``zero-point energy" error $\eta_{\text{ZPE}} =
\frac{4\pi\alpha}{\kappa} (E_R - E_T)$ and the other is the
``level-spacing error" $\eta_{\text{L}} (\beta) = \frac{2\pi
\alpha}{\kappa} [E_R(\beta+1) - E_R(\beta)] -1$ .

\begin{figure}[htb]
  \centerline{\includegraphics[width=0.7\linewidth]{zpe.png}}
  \caption{Zero point energy error.}
  \label{fig:zpe}
\end{figure}


Looking at the plot of the zero-point energy error, one can
distinguish two regimes. The first one is dominated by bandwidth
effects (for small $q$ and small $\kappa$). The other regime (for
larger $q$ and $\kappa$) is where the band is flat, but band mixing
starts to play an important role. In between these two regimes there
is a transition line along which the error goes to zero.


\begin{figure}[htb]
  \centerline{\includegraphics[width=0.7\linewidth]{psign.png}}
  \caption{Small $\kappa$ limit.}
  \label{fig:error_small_kappa}
\end{figure}


In the limit of very weak traps, $\eta_{\text{ZPE}}$ is negative.
The level spacing error seems to be dominated by band-mixing effects, due to the variation of the bandgap.

\end{document}
